\documentclass[lettersize,journal]{IEEEtran}
\usepackage{amsmath,amsfonts}
\usepackage{algorithmic}
\usepackage{algorithm}
\usepackage{array}
\usepackage[caption=false,font=normalsize,labelfont=sf,textfont=sf]{subfig}
\usepackage{textcomp}
\usepackage{stfloats}
\usepackage{url}
\usepackage{verbatim}
\usepackage{graphicx}
\usepackage{cite}
\hyphenation{op-tical net-works semi-conduc-tor IEEE-Xplore}
% updated with editorial comments 8/9/2021

\begin{document}

\title{Confiabilidade de Linhas de Transmissão\\ Utilizando Sistema Sul Brasileiro \\ com 32 Barras}

\author{Leonardo Felipe da Silva dos Santos,~\IEEEmembership{CEESP, PPGEE,~UFSM}}
        % <-this % stops a space
% \thanks{This paper was produced by the IEEE Publication Technology Group. They are in Piscataway, NJ.}% <-this % stops a space
% \thanks{Manuscript received April 19, 2021; revised August 16, 2021.}}

% The paper headers
% \markboth{Journal of \LaTeX\ Class Files,~Vol.~14, No.~8, August~2021}%
% {Shell \MakeLowercase{\textit{et al.}}: A Sample Article Using IEEEtran.cls for IEEE Journals}

% \IEEEpubid{0000--0000/00\$00.00~\copyright~2021 IEEE}
% Remember, if you use this you must call \IEEEpubidadjcol in the second
% column for its text to clear the IEEEpubid mark.

\maketitle

\begin{abstract}
This document describes the most common article elements and how to use the IEEEtran class with \LaTeX \ to produce files that are suitable for submission to the IEEE.  IEEEtran can produce conference, journal, and technical note (correspondence) papers with a suitable choice of class options. 
\end{abstract}

\begin{IEEEkeywords}
Article submission, IEEE, IEEEtran, journal, \LaTeX, paper, template, typesetting.
\end{IEEEkeywords}

\section{Introdução}
\IEEEPARstart{O}{} sistema elétrico brasileiro é constituído fundamentalmente por usinas hidrelétricas de grande porte, quais essas criam desafios para linhas de transmissão (LTs), quais hoje no Brasil o sistema em anel propõem uma segurança para o escoamento de energia e também cria um sistema de troca de energia entre as regiões, assim o sistema pode encontrar problema para distribuição de diversas cargas localizadas em locais pontuais com falta de geração ou demandas quais superam a intercambialidade de regiões.

Assim as capacidades da transmissão de energia ficam voltadas a confiabilidade do sistema elétrico de potência para escoamento dos geradores, quais o Brasil é referencia em usar hidrelétricas em sua grande maioria, normalmente localizadas na parte norte do Brasil por apresentar uma hidrologia mais favoráveis a geração hidrelétrica.


\section{Confiabilidade de Sistemas \\ Elétricos de Potência}

Qualquer sistema de potência está sujeito a falhas pontuais, tanto em equipamentos dispostos nas subestações quanto em linhas de transmissão, quais estas falhas podem comprometer a operação em parte ou todo sistema de potência, qual pode inviabilizar o fornecimento de energia em vários pontos e até mesmo para consumidores finais.

Assim a confiabilidade por meio da análise dos índices probabilísticos do sistema, combinado com julgamentos sobre critérios, 

\section{Where to Get \LaTeX \ Help --- User Groups}
The following online groups are helpful to beginning and experienced \LaTeX\ users. A search through their archives can provide many answers to common questions.

\section{Other Resources}
See for resources on formatting math into text and additional help in working with \LaTeX .

\section{Text}



\section{Some Common Elements}

\subsection{Arrays}
The {\tt{array}} environment allows you some options for matrix-like equations. You will have to manually key the fences, but there are other options for alignment of the columns and for setting horizontal and vertical rules. The argument to {\tt{array}} controls alignment and placement of vertical rules \cite{PIOTROWSKI2021} .



\section{Conclusion}
The conclusion goes here.


%{\appendices
%\section*{Proof of the First Zonklar Equation}
%Appendix one text goes here.
% You can choose not to have a title for an appendix if you want by leaving the argument blank
%\section*{Proof of the Second Zonklar Equation}
%Appendix two text goes here.}
 
\bibliography{LeonardoReferencias}
\bibliographystyle{IEEEtran}

\end{document}



